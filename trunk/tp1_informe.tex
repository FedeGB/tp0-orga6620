% ------ Clase de documento ------
\documentclass[a4paper,10pt,oneside]{article}

% ------ Paquetes ------
\usepackage{graphicx}
\usepackage[latin1]{inputenc}
\usepackage[activeacute,spanish]{babel}
\usepackage{verbatim}

% ------ Configuraci�n ------
\title{\textbf{Trabajo Pr\'{a}ctico 0:\\ Infraestructura b�sica}}

\author{    Dami�n Muszalski, \textit{Padr�n Nro. 88462}                	\\
            \texttt{ damianfiuba@gmail.com }                            	\\
            Federico Garc�a, \textit{Padr�n Nro. 93379}      			\\
            \texttt{ fedeagb@gmail.com }                       		\\
            Nicol�s Fernandez Lema, \textit{Padr�n Nro. }       		\\
            \texttt{ nicolasfernandezlema@gmail.com }                       	\\
										\\
            \normalsize{2do. Cuatrimestre de 2013}                      	\\
            \normalsize{66.20 Organizaci�n de Computadoras}             	\\
            \normalsize{Facultad de Ingenier�a, Universidad de Buenos Aires}	\\
       }
\date{10/09/2013}


\begin{document}

% Inserta el t�tulo.
\maketitle

% Quita el n�mero en la primer p�gina.
\thispagestyle{empty}

\newpage

\section{Objetivo}
El objetivo de este trabajo es familiarizarse con las herramientas que utilizaremos para emular un entorno MIPS32.


\section{Introducci\'{o}n}

En el presente trabajo pr\'{a}ctico, se deben implementar en lenguaje C una versi\'{o}n del comando \emph{rev} de UNIX. El mismo concatena y escribe en stdout el contenido de uno o mas
archivos, invirtiendo el orden de los caracteres de cada l\'{i}nea. En nuestro caso se asume que cada caracter mide 1 byte.


\section{Implementaci\'{o}n}

El programa debe leer los datos de entrada desde stdin o bien desde uno o m\'{a}s archivos. La salida del programa debe imprimirse por stdout, mientras que los errores deben imprimirse por
stderr. La ayuda y versi\'{o}n del programa puede seleccionarse mediante las opciones -h o -V respectivamente. La implementaci\'{o}n del comando debe consiste en una
funci\'{o}n con el siguiente prototipo:
int procesarArchivo(FILE* fd);

\newpage
\section{Desarrollo}
A continuaci\'{o}n se citan los pasos a que se siguieron para hacer el trabajo pr\'{a}ctico.

\subsection{Paso 1: Configuraci\'{o}n de Entorno de Desarrollo}
El primer paso fue configurar el entorno de desarrollo, de acuerdo a la gu\'{\i}a facilitada por la c\'{a}tedra.
Trabajamos con la distribuci\'{o}n Ubuntu (basada en Debian) que se puede bajar libremente de http://www.ubuntu.com/.
Se realiz\'{o} posteriormente la instalaci\'{o}n de GxEmul para emular un sistema MIPS; se utiliz\'{o} el proporcionado por la c\'{a}tedra, el cual tra\'{i}a una imagen del sistema
operativo NetBSD con algunas utilidades (compilador C, ssh, editor vi, etc).

\subsection{Paso 2: Implementaci\'{o}n del programa}
El programa se subdividi\'{o} en dos funciones:\\
- main, encargada del parseo de los par\'{a}metros de entrada.\\
- procesarArchivo, encargada de realizar la funcionalidad del comando rev.\\

El programa debe ejecutarse por l\'{i}nea de comando, donde la salida sera tambi\'{e}n por consola.

\subsubsection{Implementaci\'{o}n de las funciones}

\begin{itemize}
\item main
Es la funci\'{o}n principal, encargada de tomar los par�metros de entrada. Luego imprimir ayuda o versi\'{o}n si corresponde, o realizar llamadas sucesivas a la
funci\'{o}n procesarArchivo con el puntero a archivo que corresponda, ya sea el cual cuyo nombre se recibe como par\'{a}metro o simplemente \emph{stdin}.
		
\item procesarArchivo
Es la funci\'{o}n encargada de realizar la inversi\'{o}n de los caracteres de las l\'{i}neas de los archivos de entrada. Lee secuencialmente las l\'{i}neas reservando 1 byte
de memoria adicional con cada caracter le\'{i}do.\\
Una vez en memoria se arma un segundo arreglo de caracteres donde se copia la l\'{i}nea invirtiendola. Finalmente se imprime por pantalla esta l\'{i}nea invertida.\\
Se libera la memoria y se lee la siguiente l\'{i}nea del archivo. El ciclo corta al llegar a una se�al de EOF.

\end{itemize}
	
\subsubsection{Ingreso de par\'{a}metros}
El formato para invocar al programa es el siguiente:\\
\emph{./tp1 nombreArchivo1 nombreArchivo2 .. nombreArchivoN\\}
Donde nombreArchivoX posee el texto al cual aplicar el comando rev.


\newpage	
\section{C\'{o}digo para compilar el programa con gcc}
	
El proyecto se debe compilar tanto en el sistema Host (en nuestro caso Ubuntu) como en el Guest (NetBSD). Se dispone del mismo compilador en ambos sistemas por lo tanto
para ambos casos debemos situarnos en el directorio donde se encuentran todos los fuentes y utilizar el siguiente comando:

\texttt{gcc -std=c99 -o tp0 tp0.c}\\

Con esto se generar\'{a} un archivo ejecutable, llamado tp0.  

\newpage
\section{Ejemplos de ejecuci\'{o}n}
A continuaci\'{o}n se mostrar\'{a}n algunos ejemplos de archivos sobre los cuales se utiliza el programa realizado.\\
Considerando que los caracteres que vamos a tener en cuenta son de 1 byte realizaremos pruebas sobre archivos de texto similares al ingl\'{e}s generados aleatoreamente
utilizando la herramienta proporcionada en el sitio \\http://randomtextgenerator.com/\\
\\
\$ cat ejemplo1.txt\\ 
Turned it up should no valley cousin he.\\
Speaking numerous ask did horrible packages set. Ashamed herself has distant can studied mrs. Led therefore its middleton perpetual fulfilled provision frankness.\\
Small he drawn after among every three no. All having but you edward genius though remark one. \\
\\
Perhaps far exposed age effects. Now distrusts you her delivered applauded affection out sincerity. As tolerably recommend shameless unfeeling he objection consisted.\\
She although cheerful perceive screened throwing met not eat distance.\\
Viewing hastily or written dearest elderly up weather it as. So direction so sweetness or extremity at daughters. Provided put unpacked now but bringing.\\
\\
\$ ./tp0 ejemplo1.txt\\ 
.eh nisuoc yellav on dluohs pu ti denruT\\
.ssenknarf noisivorp dellifluf lauteprep notelddim sti erofereht deL .srm deiduts nac tnatsid sah flesreh demahsA .tes segakcap elbirroh did ksa suoremun gnikaepS\\
 .eno kramer hguoht suineg drawde uoy tub gnivah llA .on eerht yreve gnoma retfa nward eh llamS\\
\\
.detsisnoc noitcejbo eh gnileefnu sselemahs dnemmocer ylbarelot sA .ytirecnis tuo noitceffa dedualppa dereviled reh uoy stsurtsid woN .stceffe ega desopxe raf spahreP\\
.ecnatsid tae ton tem gniworht deneercs eviecrep lufreehc hguohtla ehS\\
 .gnignirb tub won dekcapnu tup dedivorP .srethguad ta ytimertxe ro ssenteews os noitcerid oS .sa ti rehtaew pu ylredle tseraed nettirw ro ylitsah gniweiV\\
 \\
 \\
\$ cat ejemplo2.txt \\
Bepaalde ik mogelijk interest gestoken in de wisselen er. Ten dan toe kinderen uitgaven stampers verhoogd. Leeningen wat uit wassching siameezen. ondernomen af verscholen en en. Drong die weg mei ploeg tabak ook. Kan bewijs deelen dan gambir midden ceylon. \\
\\
Mee geheelen wat gas kapitaal strooien kolonist mineraal. Ook erin wie maar zien weer moet ader. Aard maar nu wier de daad. Personeel levert nu om. Werkt van dit wijze buurt dagen bezet een heele. \\
\\
Afstands in is cultures reiziger de. Pahang geruineerd in plotseling. Jungles bijgang te nu beperkt op vrouwen. En doelmatige om ad traliewerk bevorderen.\\
\\
\$ cat ejemplo2.txt \textbar ./tp0 \\
.nolyec neddim ribmag nad neleed sjiweb naK .koo kabat geolp iem gew eid gnorD .ne ne nelohcsrev fa nemonredno .nezeemais gnihcssaw tiu taw negnineeL .dgoohrev srepmats nevagtiu nerednik eot nad neT .re nelessiw ed ni nekotseg tseretni kjilegom ki edlaapeB\\
\\
.eleeh nee tezeb negad truub ezjiw tid nav tkreW .mo un trevel leenosreP .daad ed reiw un raam draA .reda teom reew neiz raam eiw nire koO .laarenim tsinolok neioorts laatipak sag taw neleeheg eeM\\
\\
.neredroveb kreweilart da mo egitamleod nE .newuorv po tkrepeb un et gnagjib selgnuJ .gnilestolp ni dreeniureg gnahaP .ed regizier serutluc si ni sdnatsfA\\
\\

\section{Comparaci\'{o}n con comando rev de UNIX}
Realizamos las siguientes pruebas para comparar la salida generada por el Trabajo Pr\'{a}ctico con el comando \emph{rev} de UNIX.\\
\\
\$ ./tp0 ejemplo1.txt \textgreater ejemplo1.out\\
\$ rev ejemplo1.txt \textgreater ejemplo1.out.rev\\
\$ diff ejemplo1.out ejemplo1.out.rev\\
\\
\$ cat ejemplo2.txt \textbar ./tp0 \textgreater ejemplo2.out\\
\$ rev ejemplo2.txt \textgreater ejemplo2.out.rev\\
\$ diff ejemplo2.out ejemplo2.out.rev\\
\\
Para todos los casos no se obtuvo salida por consola por lo que ambos archivos de salida son iguales.


\newpage
\section{Conclusi\'{o}n}

\newpage
\section{Bibliograf\'{i}a}

\begin{itemize}
 \item {Material de la c\'{a}tedra}\\
 Se puede encontrar en el grupo Yahoo:\\
 \emph{http://groups.yahoo.com/neo/groups/orga-comp/files}


\end{itemize}

\newpage
\section{C\'{o}digo Fuente}
\subsection{tp0.c}
\verbatiminput{tp0.c}

\newpage
\section{C\'{o}digo MIPS generado por el compilador}
\subsection{tp0.s}
\verbatiminput{tp0.s}

\end{document}
